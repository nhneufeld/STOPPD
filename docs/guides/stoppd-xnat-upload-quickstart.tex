\documentclass[a4paper,11pt,oneside]{book}
\usepackage[top=0.35in, bottom=0.35in, left=0.5in, right=0.5in]{geometry} % margins
\usepackage{scrextend} % indenting blocks of text
\usepackage{multicol}           % column formatting
\usepackage{xcolor}             % for colors
\usepackage{graphicx}           % background image
\usepackage{tikz}               % watermark
\usepackage{transparent}        % watermark

\usepackage{datetime}

\definecolor{bg}{rgb}{0.98,0.98,0.98}

\usepackage{enumitem} % for bullets
\usepackage{mathcomp} % for bullets
\usepackage{hyperref} % for links

% Open Font
\usepackage[default,osfigures,scale=0.95]{opensans} %% Alternatively
%% use the option 'defaultsans' instead of 'default' to replace the
%% sans serif font only.
\usepackage[T1]{fontenc}

\usepackage{eso-pic}


% a set of functions
%%%%%%%%%%%%%%%%%%%%%%%%%%%%%%%%%%%%%%%%%%%%%%%%%%%%%%%%%%%%%%%%%%%%%%%%%%%%%%%% 

% formatting for headings
\newcommand\headings{
\noindent
\large
\leftskip=0.15in
\textbf
}

\newcommand\headingshack{
\noindent
\large
\leftskip=0in
\textbf
}

% formatting for titles
\newcommand\titlesL{%
    \noindent
    \small
    \leftskip=0in
    \textbf
}

\newcommand\titlesR{%
    \noindent
    \small
    \leftskip=0.15in
    \textbf
}

% modify spacing
\newcommand\spacing{
    \vspace{0.1in}
}

%formatting for steps
\newcommand\stepsL{
    \noindent
    \leftskip=0.15in
    \small
}

\newcommand\stepsR{
    \noindent
    \leftskip=0.3in
    \small
}

%formatting for naming
\newcommand\namingR{
    \small
    \noindent
    \leftskip=0.15in
    \textbf
}

%formatting for naming
\newcommand\namingL{
    \small
    \noindent
    \leftskip=0in
    \textbf
}

%formatting for namingexample
\newcommand\namingexampleR{
	\noindent
	\leftskip=0.3in
	\small
}

%formatting for namingexample
\newcommand\namingexampleL{
    \noindent
    \leftskip=0.15in
    \small
}
	
%% some preable stuff
\pagenumbering{gobble}  % rm pg. #, reset to 1
\setlist[1]{noitemsep} % sets the itemsep and parsep for lists to 0 
\begin{document}
%\AddToShipoutPicture*{\BackgroundPic} % add graphic


% header
%%%%%%%%%%%%%%%%%%%%%%%%%%%%%%%%%%%%%%%%%%%%%%%%%%%%%%%%%%%%%%%%%%%%%%%%%%%%%%%%
\fboxsep0pt
\noindent
%\colorbox{bg}{
\begin{minipage}{\textwidth}

    \vspace{0.1in} 
    \begin{multicols}{2}
        
		\topskip0pt
		\vspace*{\fill}
		\large\textbf{STOPPD XNAT Upload Quickstart Guide}
		\vspace*{\fill}

        \columnbreak
        \normalsize

        \hfill joseph@viviano.ca | joseph.viviano@camh.ca

        \hfill Center for Addiction and Mental Health, Toronto\
        
        \hfill \url{https://github.com/tigrlab/stoppd}\

    \end{multicols}
    \vspace{0in}
\end{minipage}
%}


%
%%%%%%%%%%%%%%%%%%%%%%%%%%%%%%%%%%%%%%%%%%%%%%%%%%%%%%%%%%%%%%%%%%%%%%%%%%%%%%%
\vspace{0.1in}
\begin{multicols}{2}

%
%%%%%%%%%%%%%%%%%%%%%%%%%%%%%%%%%%%%%%%%%%%%%%%%%%%%%%%%%%%%%%%%%%%%%%%%%%%%%%%
\noindent\leftskip=0in\large\textbf{ACCESSING STOPPD DATABASE}\\

\titlesL{\url{http://da55.pet.utoronto.ca:5004/spred}} \\

\titlesL{DICOM Port: 8104. Calling \& Called Title: XNAT.} \\

\stepsL{Log in using your site's username and password.} \\

\stepsL{The master project, `STOPPD01', will contain all of the data. `STOPPD02'--`STOPPD04' will contain each site's data. This allows us to keep the data organized between sites. Each site will be supplied with a username and password for accessing their `STOPPDXX' project.} \\

\stepsL{This guide is a hyper-condensed version of a more explicit guide written for the \href{https://github.com/TIGRLab/SPINS/blob/master/docs/guides/spred-upload-tutorial-v1.5.pdf?raw=true}{SPReD database, available on GitHub}. \\

%
%%%%%%%%%%%%%%%%%%%%%%%%%%%%%%%%%%%%%%%%%%%%%%%%%%%%%%%%%%%%%%%%%%%%%%%%%%%%%%%
\noindent\leftskip=0in\large\textbf{UPLOADING DATA}\\

\titlesL{DICOM Header Requirements} \\

\stepsL{The XNAT database requires some information in the DICOM header to be intact.} \

\stepsL{\begin{itemize}  
            \item{(0008,0020), (0008,0021), (0008,0022) -- Scan Date.}  
            \item{(0010,0010) -- Patient Name (e.g., STOPPD01\_MRC\_0021\_01\_01).}  
            \item{(0010,0020) -- Patient ID (e.g., STOPPD01\_MRC\_0021\_01\_01).} 
            \begin{itemize}
                \item{This is because Name \& ID become Subject \& Session, respectively, and in the SPINS projects these are identical.}
            \end{itemize}
        \end{itemize}
        }

\titlesL{Upload Overview} \\

\stepsL{Ensure your raw DICOM scans for a given participant are in each in their own folder.} \\

\stepsL{\textbf{Do not} include NIFTI or otherwise converted files in this package, but do include `.bvec' and `.bval' data.} \\

\stepsL{\textbf{Do not} include any pre-processed data from the scanner (e.g., motion corrected data, FA maps, etc.) These pollute the database and we can generate better images in post-processing.} \\

\stepsL{Package your raw DICOM folders into a appropriately named session ZIP file (.e.g., STOPPD01\_CMH\_0032\_01\_01).} \\

\stepsL{Upload your ZIP file to the database using the browser-based GUI.} \\

\titlesL{Create New Subject} \\

\stepsL{Subject IDs must be assigned by your study coordinator.} \\

\stepsL{From top of Project page, `New' $\rightarrow$ `Subject'.} \\

% \stepsL{Subject ID is mandatory, Age and Gender strongly recommended.} \\

% \stepsR{Use the Uploader Applet for automatic upload \textbf{(Not working yet!)}.} \\

%
%%%%%%%%%%%%%%%%%%%%%%%%%%%%%%%%%%%%%%%%%%%%%%%%%%%%%%%%%%%%%%%%%%%%%%%%%%%%%%%
\columnbreak

%
%%%%%%%%%%%%%%%%%%%%%%%%%%%%%%%%%%%%%%%%%%%%%%%%%%%%%%%%%%%%%%%%%%%%%%%%%%%%%%%
\titlesR{Upload DICOM Images as .zip} \\

\stepsR{From top of Subject page: `Upload' $\rightarrow$ `Images'.} \\ 

\stepsR{\begin{itemize}  
            \item{Select `Click Here' (near bottom) $\rightarrow$ `Compressed Upload' with SPINS project and destination `Prearchive'.}  
            \item{Choose a DICOM zip file, then `Begin Upload' (and wait for upload).}  
            \item{`Upload' $\rightarrow$ `Go to prearchive'. Select uploaded session, then `Archive'.}  
            \item{Check/correct subject ID, make session ID identical to subject ID, review other fields, then `Submit'.}
            \item{NB: Because each subject has exactly one session, the session should be identical to the subject name used for the .zip file.}
        \end{itemize}
        }

\stepsR{A note about manual uploads: The precise naming of the DICOM header fields will automate most of the session and scan generation process, making your life rather easy. Please see below for a naming guide.}\\


%
%%%%%%%%%%%%%%%%%%%%%%%%%%%%%%%%%%%%%%%%%%%%%%%%%%%%%%%%%%%%%%%%%%%%%%%%%%%%%%%
\titlesR{Upload Non-DICOM Data (`Tech Notes')} \\


\stepsR{\begin{itemize}

    \item{From Session page, click `MR Session' $\rightarrow$ `Manage Files'. You should see a list of the DICOM scans.}\\

    \item{Click `Add Folder'. Set `Level' to `resources' and `folder' to `notes'. Click create.}\\

    \item{Click `Upload Files', `Level' == `resources', `Folder' == `notes'. Click `Browse' and attach the notes.pdf file. Click `Upload'.}\\
    
    \end{itemize}
}

\stepsR{Ensure the uploaded data is all present in the tree.}\\

\stepsR{Now, the data will be included alongside the DICOM images in future downloads.}\\

%
%%%%%%%%%%%%%%%%%%%%%%%%%%%%%%%%%%%%%%%%%%%%%%%%%%%%%%%%%%%%%%%%%%%%%%%%%%%%%%%
\headings{GLOSSARY} \\

\stepsL{Sites:}
\stepsL{\begin{itemize}  
            \item{CMH - Centre of Addiction and Mental Health}
            \item{MAS - U Massachusetts}
            \item{NKI - Nathan Kline Institute}
            \item{PMC - Pittsburg Medical Center}
        \end{itemize}
        }

%
%%%%%%%%%%%%%%%%%%%%%%%%%%%%%%%%%%%%%%%%%%%%%%%%%%%%%%%%%%%%%%%%%%%%%%%%%%%%%%%
\headings{NAMING CONVENTIONS \& SPECIFICS}\\

\stepsL{Each subject's data is to be submitted via an appropriately named .zip file. These folders should each (only) contain a set of DICOM images from a single scanning session.} \\

%
%%%%%%%%%%%%%%%%%%%%%%%%%%%%%%%%%%%%%%%%%%%%%%%%%%%%%%%%%%%%%%%%%%%%%%%%%%%%%%%
\columnbreak
%
%%%%%%%%%%%%%%%%%%%%%%%%%%%%%%%%%%%%%%%%%%%%%%%%%%%%%%%%%%%%%%%%%%%%%%%%%%%%%%%

\namingL{Participants} \\
\namingexampleL{STOPPD01\_CMH\_0009\_01\_01}\\
\namingexampleL{[Study]\_[Site]\_[Subject]\_[Session]\_[Repeat]}\\

\stepsL{Session `01' is the first timepoint, and `02' is the one year follow-up. Repeat will typically be `01'. In the case that the participant needs to leave the scanner and return at a later time/date under a second session, repeat will be `02'.}\\

%%%%%%%%%%%%%%%%%%%%%%%%%%%%%%%%%%%%%%%%%%%%%%%%%%%%%%%%%%%%%%%%%%%%%%%%%%%%%%%
\headingshack{APPENDIX}\\

\stepsL{These steps might be required if the automatic subject/session creation with .zip upload does not function properly.} \\


\titlesL{Create New Session/Experiment} \\

\stepsL{Session IDs should be alpha-numeric starting with the participant code (`0' or `P'), and a three-digit number denoting participant number (`001').} \\

\stepsL{\textbf{A session is automatically created when a DICOM zip file is loaded, so one can normally skip this step!}} \\

\stepsL{To manually create a session (e.g. MR):}

\stepsL{\begin{itemize}
            \item{From Subject page `Actions' menu $\rightarrow$ `Add Experiment'.}
            \item{Select `MR Session'.}
            \item{Enter Session Name (and optional details).}
            \item{Delete unused scan rows (scissors).}
            \item{Enter scan number(s), type (MR), quality (usable).}
            \item{`Submit'.}
            \end{itemize}
        } \


\titlesR{Create New Scans} \\

\stepsR{\textbf{Scan(s) automatically created when a DICOM zip file is loaded, so one can normally skip this step!}} \\

\stepsR{To manually create a scan:}
\stepsR{\begin{itemize}    
            \item{From Session page `Actions' menu $\rightarrow$ `Edit' $\rightarrow$ `Add Scan'.}
            \item{Enter scan number(s), type, quality $\rightarrow$ `Submit'.}
        \end{itemize}
        } \

%
%%%%%%%%%%%%%%%%%%%%%%%%%%%%%%%%%%%%%%%%%%%%%%%%%%%%%%%%%%%%%%%%%%%%%%%%%%%%%%%
\headings{MORE HELP}\\

\small{More information is available on our public GitHub page \url{https://github.com/TIGRLab/stoppd}. This is a private repo, so you will need to contact \url{joseph@viviano.ca} to gain access. Here, you can file issues directly with us, view our wiki, and see all of the collected documentation (including this document) under /docs.}\\

\noindent\small{\textbf{Please do not put any identifying patient information here!}}

%
%%%%%%%%%%%%%%%%%%%%%%%%%%%%%%%%%%%%%%%%%%%%%%%%%%%%%%%%%%%%%%%%%%%%%%%%%%%%%%%
\end{multicols}

\center\footnotesize 
Compiled on \usdate\today. Check periodically for updates.

\end{document}
